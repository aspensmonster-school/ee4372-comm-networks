
%% stripped down version of "bare_jrnl.tex" for use in Casey's Circuits I class.
%% Original version has very good comments for use. You should check it out at
%% http://www.ieee.org/publications_standards/publications/authors/authors_journals.html
%% I run GNU/Linux so I downloaded the "Unix LaTeX2e Transactions Style File" package
%% and based my work off of the sample tex file named "bare_jrnl.tex".

% original author info below (this guy's a rockstar for making his comments so easy to use :P)
%% 2007/01/11
%% by Michael Shell
%% see http://www.michaelshell.org/
%% for current contact information.

\documentclass[journal,twocolumn]{IEEEtran}
% make sure "IEEEtran.cls" is in the path of the tex file you are working on

%graphics package for adding images
\ifCLASSINFOpdf
  \usepackage[pdftex]{graphicx}
\else
   \usepackage[dvips]{graphicx}
\fi

%math package for math equations
\usepackage[cmex10]{amsmath}

%float package for putting images where I fucking tell them to go
\usepackage{float}

%center captions under their images/figures
\usepackage{caption}

%for sourcecode
\usepackage{listings}
\lstset{breaklines=true,language=gnuplot,basicstyle=\scriptsize,showspaces=false,showstringspaces=false}

%For hyperlinks
\usepackage{hyperref}

\begin{document}

% paper title
% can use linebreaks \\ within to get better formatting as desired
\title{Project 3: Collection and Analysis of Wireless Signals}
\author{Preston~Maness}

% header
\markboth{Texas State University, Dr. Hudson, EE4372, Spring 2014}%
{}

% make the title area
\maketitle

% Give the abstract of your lab here
\begin{abstract}
A route around campus was taken on foot in order to sample cell phone, GPS, 
and WiFi signals. An Android smartphone with the RF Signal Tracker 
app was used to sample and store the various data associated with each type 
of wireless signal. Analysis was then performed on signal strength over 
time and location for each type of signal. Additionally, wireless network
SSID and BSSID characteristics were correlated against location on the 
Texas State University San Marcos campus.

\end{abstract}

% Split your lab report into sections by calling \section{Section name}
\section{Sampling Methodology and Route}
\IEEEPARstart{T}{his} project makes use of an app titled ``RF Signal Tracker'' 
by developer Ken Hunt in the Google Play app store for Android smart phones. 
The particular device utilized was a Samsung Galaxy S Relay 4G, model string 
SGH-T699. A path was walked around campus, with this device running this app,
to sample cellular signals, GPS signals, and WiFi signals at regular intervals
and store the results in a CSV file. Analysis was then performed on the data.

The route is detailed in figure~\ref{fig_path} as the red line over the Google
Maps Satellite image. The path started at the RFM building, and ended at the
commuter parking lot near the soccer fields.

\begin{figure*}
\begin{center}
\includegraphics[scale=0.4]{path-data/Path-Taken-Starting-At-Mitte.png}
\caption{Path Taken Around Campus}
\label{fig_path}
\end{center}
\end{figure*}

\section{Data}

The data is kept in two principal csv files on a git repository located at 
the following URLs:

\begin{itemize}
\item
Cell and GPS data: \url{https://raw.githubusercontent.com/aggroskater/ee4372-comm-networks/master/project-3-wireless-signal-sampling/cell-data/export_140321144244.csv}
\item
WiFi data: \url{https://raw.githubusercontent.com/aggroskater/ee4372-comm-networks/master/project-3-wireless-signal-sampling/wifi-data/wifi_140403220558.csv}
\end{itemize}

The data collected Cell and GPS data every 5 to 6 seconds, and WiFi data every 
7 or 8 seconds. It is important to note that it appears as if the sampling 
software was unable to keep up with the polling requests for at least the 
first minute for the WiFi data. There are multiple sets of WiFi data that have
the same timestamp, but different GPS coordinates during this first minute. The
overal trends gleaned from the data, however, do not suffer unnecessarily from
this glitch.

Numerous gnuplot scripts were utilized to plot this data in different ways. 
Along the way, various ``helper'' CSVs were generated to ease plotting with 
gnuplot. These helper CSVs may all be found in the repository, as can  the 
gnuplot scripts (the scripts may also be found in Appendix A of this report).

\section{Cellular Data Analysis}

Interior crocodile alligator. I drive a chevrolet movie theatre.

\section{WiFi Data Analysis}

Much analysis. Many data. So plot. Wow.

\section{Conclusion}

Lol conclusions are cool.

\appendix[Gnuplot Scripts]

This appendix holds numerous gnuplot scripts utilized in the 
generation of the plots found in this report.

\begin{lstlisting}
asdf
\end{lstlisting}

\end{document}
