
%% stripped down version of "bare_jrnl.tex" for use in Casey's Circuits I class.
%% Original version has very good comments for use. You should check it out at
%% http://www.ieee.org/publications_standards/publications/authors/authors_journals.html
%% I run GNU/Linux so I downloaded the "Unix LaTeX2e Transactions Style File" package
%% and based my work off of the sample tex file named "bare_jrnl.tex".

% original author info below (this guy's a rockstar for making his comments so easy to use :P)
%% 2007/01/11
%% by Michael Shell
%% see http://www.michaelshell.org/
%% for current contact information.

\documentclass[journal,twocolumn]{IEEEtran}
% make sure "IEEEtran.cls" is in the path of the tex file you are working on

%graphics package for adding images
\ifCLASSINFOpdf
  \usepackage[pdftex]{graphicx}
\else
   \usepackage[dvips]{graphicx}
\fi

%math package for math equations
\usepackage[cmex10]{amsmath}

%float package for putting images where I fucking tell them to go
\usepackage{float}

%for sourcecode
\usepackage{listings}
\lstset{breaklines=true,language=gnuplot,basicstyle=\scriptsize,showspaces=false,showstringspaces=false}

%For hyperlinks
\usepackage{hyperref}

\begin{document}

% paper title
% can use linebreaks \\ within to get better formatting as desired
\title{Project 3: Collection and Analysis of Wireless Signals}
\author{Preston~Maness}

% header
\markboth{Texas State University, Dr. Hudson, EE4372, Spring 2014}%
{}

% make the title area
\maketitle

% Give the abstract of your lab here
\begin{abstract}
A route around campus was taken on foot in order to sample cell phone, GPS, 
and WiFi signals. An Android smartphone with the RF Signal Tracker 
app was used to sample and store the various data associated with each type 
of wireless signal. Analysis was then performed on signal strength over 
time and location for each type of signal. Additionally, wireless network
SSID and BSSID characteristics were correlated against location on the 
Texas State University San Marcos campus.

\end{abstract}

% Split your lab report into sections by calling \section{Section name}
\section{Introduction}
\IEEEPARstart{T}{his} project... 

%\begin{figure}[h!]
%\centering
%\includegraphics[scale=.35]{schematic.png}
%\label{fig_schem}
%\end{figure}

\section{Another section...}

blah blah blah...

\begin{tabular}{|c|c|c|c|}
\hline
Value & Expected & Measured & \% error \\
\hline
$\omega_{o}$ (Hz) & 440 & NA & NA\\
\hline
phase (deg) & 45 & NA & NA\\
\hline
\end{tabular}

\section{Conclusion}

blah blah blah.

\end{document}
